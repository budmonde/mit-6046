\documentclass{6046}

\author{Budmonde Duinkharjav}
\problem{3-2}
% \problem{A-B} means Problem Set A, Problem B.
\collab{none}
% or give names, e.g., \collab{Alyssa P. Hacker and A. Student}

\begin{document}

\section{Part a}
\subsection{Question 1}
Using the equations in the question statement:
\begin{eqnarray*}
    g_K &=& \min_{0 \leq k \leq K} ||\nabla f(x^k)||\\
    &\leq& \frac{1}{K+1} \sum^K_{k = 0} ||\nabla f(x^k)||.
\end{eqnarray*}
By AMQM:
\begin{eqnarray*}
    g_K &\leq& \sqrt{\frac{1}{K+1} \sum^K_{k = 0} ||\nabla f(x^k)||^2}.\\
    &=& \sqrt{\frac{2L}{K+1}} \sqrt{\sum^K_{k=0} [f(x^k) - f(x^{k+1})]}\\
    &\leq& \sqrt{\frac{2L}{K+1}} \sqrt{f(x^0) - f(x^*)}.
\end{eqnarray*}
This is true because $f(x^*) \leq f(x^{K+1})$. Hence we are done.
\subsection{Question 2}
If the $g_K$ goes below $\epsilon$, the algorithm terminates. Hence this inequality holds true. Finding $K$ from there gives
\begin{eqnarray*}
     \epsilon &\leq& \sqrt{\frac{2L}{K+1}} \sqrt{f(x^0) - f(x^*)}\\
     K &\leq& \frac{1}{\epsilon^2} [2L (f(x^0) - f(x^*)) + 1]
\end{eqnarray*}
Hence we are done.
\section{Part b}
Let's consider the convex function $f(x) = x^2$. If we choose $x^0 = 1$ and $L = 1$ we get:
\begin{eqnarray*}
     f(x^1) &=& 1 - 1 \times 2 = -1\\
     f(x^2) &=& -1 + 1 \times 2 = 1\\
     ...\\
     ...\\
\end{eqnarray*}
Hence the recursion oscillates indefinitely.
\end{document}

