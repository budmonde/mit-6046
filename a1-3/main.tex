\documentclass{6046}

\author{Budmonde Duinkharjav}
\problem{1-3}
% \problem{A-B} means Problem Set A, Problem B.
\collab{Danny Tang, Courtney Guo}
% or give names, e.g., \collab{Alyssa P. Hacker and A. Student}

\begin{document}
\section{Part a}
\subsection{Question 1}
Since we are given $N^2$ constants $S(w^i,w^j)$ for $i,j \in (0,N-1)$, to prove that this indeed will be sufficient to find all coefficients $s_{k,l}$ for $k,l \in (0,N-1)$ we must show that we can find a system of linear equations with a unique solution. For the following matrix equation, we can see that each row evaluates to one of the constants $S(w^i,w^j)$.
\[
\begin{bmatrix}
    &&&&\\
    &&&&\\
    &&&&\\
    &&w^{ik + jl}&&\\
    &&&&\\
    &&&&\\
    &&&&\\
\end{bmatrix}
\begin{bmatrix}
    s_{0,0}\\
    s_{0,1}\\
    \vdots\\
    s_{0,N-1}\\
    s_{1,0}\\
    \vdots\\
    s_{N-1, N-1}
\end{bmatrix}
=
\begin{bmatrix}
    S(w^0,w^0)\\
    S(w^0,w^1)\\
    \vdots\\
    S(w^0,w^{N-1})\\
    S(w^1,w^0)\\
    \vdots\\
    S(w^{N-1},w^{N-1})
\end{bmatrix}
\]

Let the matrix $\mathbf{M} = (w^{ik +jl})$ with $row = iN + j$ and $col = kN + l$. Then we claim that $\mathbf{M}^{-1} = \frac{1}{N^2}(w^{-ik-jl})$. We support this claim by showing that $(w^{ik +jl}) (w^{-ik-jl}) = N^2 \mathbf{I}$ where $\mathbf{I}$ is the identity matrix.

\begin{eqnarray*}
    \lambda_{row, col} &=& \sum_{m,n = 0}^{N-1} w^{im+jn} \times w^{-mk-nl}\\
    &=& \sum_{m = 0}^{N-1} w^{m(i-k)} \sum_{n=0}^{N-1} w^{n(j-l)}\\
    &=& N^2 \delta_{ik} \delta_{jl}
\end{eqnarray*}

where $\delta_{ij}$ is the Kronecker delta function. For the last result, we use the same analysis as we did in regular FFT. As such we now have an equation which solves for a unique $S(x,y)$ equation.

\[
\begin{bmatrix}
    &&&&\\
    &&&&\\
    &&&&\\
    &&w^{-ik - jl}&&\\
    &&&&\\
    &&&&\\
    &&&&\\
\end{bmatrix}
\begin{bmatrix}
    S(w^0,w^0)\\
    S(w^0,w^1)\\
    \vdots\\
    S(w^0,w^{N-1})\\
    S(w^1,w^0)\\
    \vdots\\
    S(w^{N-1},w^{N-1})
\end{bmatrix}
=
\begin{bmatrix}
    s_{0,0}\\
    s_{0,1}\\
    \vdots\\
    s_{0,N-1}\\
    s_{1,0}\\
    \vdots\\
    s_{N-1, N-1}
\end{bmatrix}
\]

\subsection{Question 2}
To evaluate $P(x,y)$ for all values of $S_N$, we will first note that we can express $P(x,y)$ as

\begin{equation*}
    P(x,y) = P_0(x) + P_1(x) y + P_2(x) y^2 + ... + P_{N-1}(x) y^{N-1}.
\end{equation*}

From here we see that we can identify $N$ polynomials of $x$ each of degree of $N$. Using FFT, we can calculate each polynomial for $N$ different values in $O(N^2 \log N)$ time. Then, again using FFT, we can calculate $P(x,y)$ for every $N$ values of $y$ in $O(N^2 \log N)$ time since we precomputed all $P_i(x)$ for all values of $x$.\\

In the above proof of evaluating points given the coefficients we effectively evaluated the first matrix in question 1. Now our objective is to evaluate the second matrix from question 1 in $O(N^2 \log N)$ time. Since both matricies look exactly the same, apart from the fact that the values are conjugates of the original values, we can use the same technique as above to calculate the coefficients in $O(N^2 \log N)$ time.


\subsection{Question 3}
Using the results from question 2, we are able to find all values of $P(x,y)$ and $Q(x,y)$ for all $(x,y) \in S$. We can find these values in $O(N^2 \log N)$ time. Then, we are able to multiply all the corresponding $P(x,y)$ with $Q(x,y)$ in $O(N^2)$ time. Finally having found all the point evaluations, we can recover the coeffients using the results found above in $O(N^2 \log N)$ time totalling in a runtime of $O(N^2 \log N)$.

\section{Part b}
\subsection{Question 1}
Using the formula given in the problem statement we find the following.

\begin{eqnarray*}
    p(x) &=& t(x) (x - a_0) + r(x)\\
    p(a_0) &=& t(a_0) (a_0 - a_0) + r(a_0)\\
    &=& r(a_0).
\end{eqnarray*}

\subsection{Question 2}
For a given polynomial $p(x)$, we can do the following:
\begin{eqnarray*}
    p(x) &=& t(x) (x - a_0) (x - a_1) + r(x)\\
    r(x) &=& t(x) (x - a_0) + z(x)\\
    z(a_0) &=& r(a_0) = p(a_0)
\end{eqnarray*}

Here, we see that if we recurrently $\mod$ by the same $(x - a)$, we can recover the original value $p(a)$ would have from a polynomial with a much smaller degree. We will use this idea to use D\&C to find $p(a_0)...p(a_{n-1})$.\\

At the start we will have the set $S$ have $n$ elements. We will split them in half and for each half we will evaluate $p(x) \mod (x-a_0)(x-a_1)...(x-a_(n/2-1))$. This will give us a remainder $r(x)$ which we will use for the next level recursive case we will evaluate $r(x) \mod (x-a_0)(x-a_1)...(x-a_(n/4-1)$. Keep in mind that we are also calculating the $\mod$ for the other ranges of $a_i$ in their respective branches of the recursion tree. The base case of this problem is to have $p(x) = t(x) (x - a_i) + r(x)$ such that $r(x)$ is of degree $1$. In this case, $p(a_i) = r(a_i)$ can be calculated in $O(1)$ time.\\

Once all base cases terminate, we will have found all $p(a_i)$. The recursive formula for the problem is
\begin{eqnarray*}
    T(n) &=& 2T(n/2) + O(n \log n)\\
    T(n) &=& n \log^2 n.
\end{eqnarray*}


\end{document}

